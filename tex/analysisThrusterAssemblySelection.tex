\documentclass[../main.tex]{subfiles}
%!TEX root = ./analysisThrusterAssemblySelection.tex
\graphicspath {{../}}

\begin{document}
\subsection{Motor and Propeller Selection} \label{motorSelect}

\begin{figure}[H]
	\centering
	\includegraphics[width=0.7\linewidth]{img/paramaterization/motorPropellerChoice.pdf}
	\caption{Parametrization Outline For Motor and Propeller Selection}
	\label{fig:motorOutline}
\end{figure}

The selection of the motor and propeller relies on experimental data. This was sourced from Fly Brushless \cite{motPropData} and the data is used to get realistic values based on the required speed. To narrow the selection of data, propellers were only chosen from APC (propeller manufacturer), given the quality and quantity of data on Fly Brushless. Roughly 2000 points of data were collected, then reduced to 60 relevant cases to improve computation time. The excluded cases would never be chosen by the program, therefore they were emitted.\\

The selection works by filtering the experimental data to get motors that fit within the power limit set by the program. From this, the unique pitch and diameter combinations are analysed to compute the required rotational speed to achieve the required speed. The required rotational speed is found iteratively, increasing each time until the thrust curve intersects the drag curve above the required speed. The analysis for the thrust calculation is found in Appendix \ref{appendix:thrust} and the drag analysis is found in Section \ref{drag}. To increase the rotational speed guess of each iteration, an equation is used to approximate the guess based on the zero thrust speed (speed at which thrust force is no longer generated). The equation relates the maximum RPM ($nm$) to the zero thrust speed $V_{zero}$, pitch $P$ and diameter of propeller $D$:

\begin{equation}
nm = \frac{V_{}zero}{0.2D+0.74P}
\end{equation}

For each iteration, the speed is increased by $1m/s$ scaling the RPM proportional until the required speed is met. These results are then compared to the experimental data and the data point which requires the lowest power is selected. If there is no matching experimental data, the program is looped with a lower required speed. The code for this selection can be seen in Section \ref{code:motorSelect}.	

\subsection{Battery Selection} \label{batterySelect}

\begin{figure}[H]
	\centering
	\includegraphics[width=0.95\linewidth]{img/paramaterization/batte
		ryChoice.pdf}
	\caption{Parametrization Outline For Battery Selection}
	\label{fig:batteryOutline}
\end{figure}

The battery selection follows the propeller and motor selection since power usage of the motor is a required input. Similar to the motor and propeller, the battery selection relies on real battery data to get weights and dimensions of batteries. This data was taken from Hobby King \cite{Hobbyking}  by arbitrarily picking Lithium Polymer batteries light in weight. A variety of batteries were chosen with a different number of cells to accommodate the different motor sizes and varying required flight times.\\

The data is filtered to remove batteries that are too heavy, do not have enough discharge, and/or do not have the required voltage. Using the inputs to the function, the required carrying capacity is calculated. This is done using Equation \ref{eqn:battCap}, where capacity is in $mAh$, motor amperage ($A$) is in amps, and time ($t$) is in hours.

\begin{equation} \label{eqn:battCap}
\text{Capacity} = At*1000;
\end{equation}

This calculated value is then compared to the data and the lightest matching battery is picked. If no batteries meet the requirements, the required time is reduced and the capacity is recalculated until it matches a battery. The chosen battery information is used to get the weight of the thruster assembly and parametrise the battery enclosure. The code for this can be seen in Section \ref{code:batterySelect}. It is important to note that this flight time is only valid for just the thruster motor. The BESC, receiver, and vector thrusting motor will all reduce this life depending on their need.

\end{document}