\documentclass[../main.tex]{subfiles}
%!TEX root = ./analysisSystemModelling.tex
\graphicspath {{../}}

\begin{document}
	
\subsection{Gondola Forces} \label{gondForces}
In order to complete the analysis on the gondola there are numerous forces and reaction that need to be computed. Reaction forces are solved using the gondolaForces code REF??? in conjunction with forceSolver which is explain in section REF?? and  the rotate code REF???. The gondolaForces code required the inputs of all the known forces acting on the gondola and their positions including the drive force of the friction wheel motor, the spring force acting on the friction wheel, the maximum thruster acceleration, the weight of each gondola car, and the linear actuator holding force. The code also requires the pitch angle of the airship ($\phi$), the thruster angle ($\beta$), the angle between the gondola cars($\Theta$), the position of each of the bearing arm reactions. All of the angles mentioned above are with reference to the top surface of the rear gondola. The code will output acceleration of the gondola, The maximum force applied the bearing arms and the braking force if the linear actuator is applied. \\

the coordinate system in which all the forces are referenced is based on the rear gondola (gondola 1) as seen in FIG ??. The code first takes the drive force and the spring force both of which are acting on the friction wheels and rotates them into the coordinated system of the rear gondola based on the 45 degree angle of the torsion hinge. The spring and drive force acting on the front gondola (gondola 2) are again rotated based on the angle between the gondolas ($\Theta$). Then the code takes the inputed maximum thrust acceleration, rotates it based on the thrust angle ($\beta$) multiplies it by the mass of each gondola car and applies it to the center of mass of the corresponding gondola car. The weight of each car is then based on the pitch angle ($\phi$) is then also applied to the center of mass of each gondola car. For the case  


\subsection{Loading Scenarios} \label{loadingScenarios}
\subsubsection*{Maximum Required Gondola Drive Force}
The following scenario shown in FIG??? is the one that would require the greatest drive force from the gondola friction wheel motors. The assumptions made for this scenario are extremely conservative in that the scenario is extremely unlikely to actually occur but would result in considerably greater forces acting against the gondola motors than any more likely situation. The scenario involves the airship pitching straight upwards (pitch angle $\phi$ of 90 degrees),thrusting straight upwards at full thrust (thrust angle $\beta$ of 0 degrees) and the gondola is on the straight section of the keel (gondola angle $\Theta$ of 0) driving up towards the curved section of the keel.

\subsubsection*{Maximum Gondola Bearing Arm Forces}
The scenario shown in FIG??? results in the largest fores being applied to the gondola bearing arms. The scenario involves the linear actuator being applied in order to hold the position of the gondola on the straight part of the keel. The airship has a pitch angle of 0 and the thrusters is  angle is 90 degrees straight up. This loading scenario results in the largest force  being applied to the bearing arms.

\end{document}