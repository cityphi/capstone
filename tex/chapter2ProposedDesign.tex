\documentclass[../main.tex]{subfiles}

\begin{document}
\chapter{Proposed Design}
Introduction Paragraph
\section{Gondola}
The gondola sub-assembly is the most distinguishable feature of the airship design. The gondola moves along the keel in order to rapidly pitch the airship. Key components can be seen in FIGURE???????. Movement along the gondola is driven by two friction wheels which interface with the cover on the keel through the friction between the two surfaces. Bearings are used as wheel to hold the gondola on top of the keel, while the friction wheel are in contact with the bottom edge of the keel. The gondola is comprised of two half sections which each house components. Joining the two sections is a hinge, that allows the gondola to bend in the centre. The bend in the middle allows the gondola to overcome more a significant turning radius compared to being fixed in a straight line. Similarly to curve fitting, the more points that are used, the better fit there is. Braking of the gondola is accomplished by turning the friction wheels in the opposite direction and the position is held by a linear actuator. The gondola has been designed to be resistant to small amounts of water, considering the gondola will always be shielded by the envelope, total waterproofing is not required. The gondola section are made using high quality 3D prints, given their complex geometries and relatively small size.
\\
\subsection{Friction Wheel}
The friction wheel is made from 55D rubber so it can deform when it is placed on the keel. To ensure the pressure on the keel, the friction wheel assembly is mounted on a bracket equipped with a torsion spring, as seen in FIGURE???????. The motor used can be applied in both directions, making it useful as a brake also. A magnetic encoder can be seen in FIGURE????, allowing for accurate positioning of the gondola sections, independently allowing for redundancy. The motor is mounted to the bracket using two nuts and two bolts. The bracket is mounted to the gondola using 2 nuts, 2 bolts and 2 washers.
\\
\subsection{Bearings}
Bearings are placed on top of the keel to support the weight of the gondola and counter the force of the friction wheel. The bearings act as wheels turning and slipping along the keel. The bearings are mounted using a snap fit design where the bearing is held in place by a diameter slightly larger than its own. The bearing is placed over the piece, deflecting it until the bearing can slip past the over hand where the snap fit piece elastically snaps back into its original position, securely holding the bearing.
\\
\subsection{Linear Actuator}
A linear actuator is housed on a section of the gondola. The linear actuator is to be activated when the gondola is not moving to hold its position. This particular linear actuator has a holding force of 45N without constant power. It is also water resistant to the IP54 standard. See FIGURE???????
\subsection{Door and Components}
The gondola features a sliding door, to ease the process of component charging and changing. The door relies upon an interference fit to stay in place and components are placed on top of it once the door is half closed. See FIGURE????? (selective cross-section).
\\
\end{document}