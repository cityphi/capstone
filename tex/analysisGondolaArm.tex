\documentclass[../main.tex]{subfiles}

\begin{document}
\section{Gondola Arm} \label{bearingArm}
The failure of the gondola arm will be analysed in a very similar fashion to the gondola arm deflection. Once again, Figure \ref{fig:deflection} is used as the basis for the analysis. Instead of deflection however, stress at the inner corner of the arm is found as:
\begin{align}
	\sigma _{Gondola Arm} = \sigma _{axial} + \sigma _{bending force} + \sigma _{bending moment} \\ \label{armStress}
	\sigma _{Gondola Arm}  = \left(\dfrac{F_{NB_{z}}}{A}\right)\hat{k} + \left(\dfrac{F_{NB_{y}}l_{arm}c}{I}  + \dfrac{M_{NB}c}{I} \right) \hat{j}
\end{align}

These stresses are converted to principle stresses (as shown in Appendix \ref{appendix:cauchy}). These principle stresses are then used to determine the safety factor by Brittle Mohr-Coulomb Theory \cite[227]{shigley}.

Since $\sigma _a > \sigma _b > 0$,

\begin{equation}
	\eta = \dfrac{S_{ut}}{\sigma _a} \Rightarrow 1.5 \geq \dfrac{S_{ut}}{\sigma _a}
\end{equation}

\end{document}