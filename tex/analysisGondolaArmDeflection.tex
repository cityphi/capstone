\documentclass[../main.tex]{subfiles}
%!TEX root = ./analysisGondolaArmDeflection.tex
\graphicspath {{../}}

\begin{document}
\section{Gondola Arm Deflection} \label{gondDeflection}
To ensure that the gondola will not fall off of the keel during operation, a deflection calculation is computed on the gondola arm. The maximum deflection of the gondola arm is modelled in a similar fashion to the Gondola Arm Stress Analysis in Section \ref{bearingArm}.

The deflection will be calculated using simple beam equations. The force $F_{NB}$ is resolved into $y$ and $z$ components. The deflection is then computed in three separate parts, as shown below:

\begin{align}
	\delta _{Gondola Arm} = \delta _{axial} + \delta _{bending force} + \delta _{bending moment} \\ \label{armDeflection}
	\delta _{Gondola Arm}  = \left(\dfrac{F_{NB_{z}}l_{arm}}{AE}\right)\hat{k} + \left(\dfrac{F_{NB_{y}}l_{arm}^3}{3EI}  + \dfrac{M_{NB}l_{arm}^2}{2EI} \right) \hat{j}
\end{align}

The faliure possibility here would be for the arm to deflect enough that the gondola falls of the keel. This occurs when the total deflection $\delta$ is larger than $0.5cm$, which is half of the width of the keel face. Since both arms can deflect at the same time, they can be combined to reach 0.5cm. Therefore it is required that the result of Equation \ref{armDeflection} be less than $0.25cm$. Therefore the failure criteria is:

\begin{multline}
	0.25cm \geq \sqrt{\left(\dfrac{F_{NB_{z}}l_{arm}}{AE}\right)^2 + \left(\dfrac{F_{NB_{y}}l_{arm}^3}{3EI}  + \dfrac{M_{NB}l_{arm}^2}{2EI} \right)^2} =\\ \sqrt{\left(\dfrac{26.6082N*0.036m}{2.874*10^{-5}m^2*1.1380*10^9Pa}\right)^2 + \left(\dfrac{-26.6082N*0.036m^3}{3*1.1380*10^9Pa*7.0686*10^{-6}m^4}  + \dfrac{2.3823Nm*0.036m^2}{21.1380*10^9Pa*7.0686*10^{-6}m^4} \right)^2}\\=0.1931mm
\end{multline}

Therefore the gondola arms have no risk of deflecting enough to cause the gondola to fall off.

\end{document}