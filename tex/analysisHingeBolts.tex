\documentclass[../main.tex]{subfiles}

\begin{document}
\subsection{Bolt Compression Force} \label{compressive}
To fasten the metal hinge to the plastic gondola body, bolts will be used. The reason for this is that they use compression forces to join the two pieces, without requiring threading. Threading a screw into plastic would require female threads in 3D printed plastic, which is inherently bad design and would almost surely fail. \\

The main mode of failure for the bolts will not be for the bolt itself, as the forces involved in this design will surely be much less than the yield strength of any potential bolt used. The true concern is the plastic at the interface of the bolt being crushed. \\

To quantify this failure mode, the bolt tension required to resist the forces parallel to the surface of the hinge must be computed. The forces parallel to the bolt are shown in Figure \ref{fig:BoltTop}. They are the same forces as those shown in other hinge analysis figures, such as Figure \ref{fig:Gondola2Top}. The resultant forces in the XY plane are determined as 

\begin{displaymath}
F_{bshear} = \sqrt[]{F_{bx'}^2 + F_{by}^2}
\end{displaymath}

The only reaction resisting forces in the XY plane will be the friction between the washer and the plastic. Therefore, $F_f > F_{bshear}$. It is known that $F_f = \mu N$. In this case, the normal force is provided by the bolt tension $F_{bolt}$ shown in Figure \ref{fig:BoltSide}. This implies that $F_f = \mu F_{bolt}$. Therefore,

\begin{displaymath}
\mu F_{bolt} > F_{bshear}
\end{displaymath}

The required bolt force to ensure no slipping due to shear forces is therefore

\begin{equation}
F_{bolt} = \dfrac{\sqrt[]{F_{bx'}^2 + F_{by}^2}}{\mu}
\end{equation}

$\mu $ is estimated as the coefficient of friction between polyethylene and steel, which is 0.2 \cite{Friction}. \\

Once $F_{bolt}$ is known, the compressive stress of the washer on the plastic gondola body can be determined. This is the critical design factor. If the compressive stress from the washer is too high it will crush the plastic underneath it.\\

The compressive yield strength ($S_{compressive}$) of Nylon 126 (a similar material) is found to be 55 MPa \cite{Compressive}. The compressive stress on the gondola by the washer is found to be 

\begin{align*}
\sigma _{washer} &= \dfrac{F_{bolt}}{A_{washer}}
\end{align*}
\begin{align}
\sigma _{washer} &= \dfrac{F_{bolt}}{\pi (r_o - r_i)^2}
\end{align}

The safety factor is then found by

\begin{displaymath}
\eta = \dfrac{S_{compressive}}{\sigma _{washer}}
\end{displaymath}
\end{document}