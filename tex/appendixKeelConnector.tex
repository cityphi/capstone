\documentclass[../main.tex]{subfiles}
%!TEX root = ./appendixKeelConnector.tex
\graphicspath {{../}}

\begin{document}
\section{Keel Connector Diamond}
This is an extension of the keel analysis Section \ref{connector}. The diamond part of the connector is at a fixed size since any changes would cause it to interfere with the wheels on the gondola. This section is just to verify that it would not be the failure point. It is made from the same piece of aluminium as the square part, however it is much larger and benefits from having the keel keeping it in compression. The worst case scenario is the same as the one done for buckling shown in Figure \ref{fig:armConnectorBuck}. The piece undergoes shear stress at the centre of the piece, given by Equation \ref{eqn:keelConnectorTau}. Where $V$ is the force acting down from the arm, $Q$ is the first moment of area, $I$ is the moment of inertia, and $b$ is the width of the diamond.

\begin{equation} \label{eqn:keelConnectorTau}
\tau_{yz} = \frac{VQ}{Ib}
\end{equation}

To get the safety factor the primary stresses have to be found and then von Mises \cite[216]{shigley} can be applied. The von Mises equation is shown in Equation \ref{eqn:vonMisesKeel} and the safety factor is calculated using \ref{eqn:safetyFactorKeel}

\begin{equation} \label{eqn:vonMisesKeel}
\sigma' = \frac{1}{2}\sqrt{(\sigma_1 - \sigma_2)^2 + (\sigma_2 - \sigma_3)^2 + (\sigma_3 - \sigma_1)^2)}
\end{equation}

\begin{equation} \label{eqn:safetyFactorKeel}
\eta = \frac{\sigma_{yield}}{\sigma'}
\end{equation}

\end{document}
